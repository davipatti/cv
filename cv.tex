% LaTeX source of my resume
% =========================

% Commented for easy reuse... ;)

% See the `README.md` file for more info.

% This file is licensed under the CC-NC-ND Creative Commons license.


% Start a document with the here given default font size and paper size.
\documentclass[10pt,a4paper]{article}

% Set the page margins.
\usepackage[a4paper,margin=0.75in]{geometry}

% For publications
\usepackage[square,numbers]{natbib}
\bibliographystyle{abbrvnat}

% Setup the language.
\usepackage[english]{babel}
\hyphenation{Some-long-word}

% Makes resume-specific commands available.
\usepackage{resume}

\begin{document}  % begin the content of the document
\sloppy  % this to relax whitespacing in favour of straight margins

% title on top of the document
\maintitle{David Pattinson}{September 19, 1989}{Last update on \today}

\nobreakvspace{0.3em}  % add some page break averse vertical spacing

% \noindent prevents paragraph's first lines from indenting
% \mbox is used to obfuscate the email address
% \sbull is a spaced bullet
% \href well..
% \\ breaks the line into a new paragraph
\noindent\href{mailto:david.pattinson@wisc.edu}{david.pattinson\mbox{}@\mbox{}wisc.edu}\sbull
\textsmaller{+1} (516) 413-4078\sbull
\href{https://github.com/davipatti}{github.com/davipatti}\sbull
\href{http://linkedin.com/in/davipatti}{linkedin.com/in/davipatti}
\\
Influenza Research Institute, 575 Science Dr., Madison, WI 53711, USA

\spacedhrule{0.9em}{-0.4em}  % a horizontal line with some vertical spacing before and after

\roottitle{Summary}  % a root section title

% \vspace{-1.3em}  % some vertical spacing

\noindent \emph{I am a computational scientist interested in understanding and
  predicting the antigenic evolution of seasonal influenza viruses, tracking
  population immune responses to SARS-CoV-2, Bayesian inference, and clean,
  reproducible research software practices. I use antigenic cartography,
  antibody landscapes, phylogenetics, computational structural biology, Bayesian
  inference and data visualisation.}


% \begin{multicols}{2}  % open a multicolumn environment

%   \noindent \emph{
%     I am a computational scientist interested in understanding
%     and predicting, the antigenic evolution of seasonal influenza viruses and
%     tracking population immune responses to SARS-CoV-2. I use antigenic
%     cartography, antibody landscapes, phylogenetics, computational structural
%     biology, Bayesian inference and data visualisation. } \\
%   \\
%   % At the age of seven (1989) Cies wrote his first lines of code in a \acr{LOGO}-like language on an \acr{MSX} (pre-\acr{PC}).  Two years later he attended a conference on an emerging new technology, the Internet, at the Erasmus University from which he would graduate 16 years later (2000) with a degree in Business Computer Science.
%   % After being introduced to the open source movement in 1997, he taught himself a variety of skills including system administration and programming (Bash, Python, Ruby \& \CPP).  By 2002 he got his pet project \acr{KT}urtle ---a zero-entry-barrier programming environment--- included in \acr{KDE}'s \emph{edu} module, and thereby in almost every Linux distribution.

%   % Extensively travelled Europe and Asia after graduating. On return his professional life became a mix of hands-on work in startups and web software agencies, while settling as a husband and father in personal life.

% \end{multicols}


\spacedhrule{0.9em}{-0.4em}

\roottitle{Experience}

\headedsection
{Influenza Research Institute, {\href{https://www.wisc.edu/}{University of Wisconsin--Madison}}}
{\textsc{Madison WI, USA}} {%
  \headedsubsection
  {Scientist}
  {Mar \apo22 -- present}
  {

    \bodytext{ I was promoted to Scientist two years after joining Prof.
      Kawaoka's laboratory having provided expertise in computational analyses
      to the group (including antigenic cartography, antibody landscapes,
      phylogenetics, computational structural biology, next-generation
      sequencing analysis). I teach in small groups, write bespoke analysis
      pipelines, and continue my own research projects.}

  }

  \headedsubsection
  {Postdoctoral Research Associate}
  {Nov \apo19 -- Mar \apo22}
  {

    \bodytext{ I joined Prof.\@ Yoshihiro Kawaoka's laboratory to primarily work
      on using computational structural biology to understand mechanisms of
      antigenic change. The COVID-19 pandemic prompted a switch to studying
      SARS-CoV-2 serology. I help plan, design experiments, and conduct analyses
      for a prospective serosurveillance project in collaboration with
        {\href{https://www.marshfieldresearch.org/}{Marshfield Clinic Research
            Institute}} and the {\href{https://www.cdc.gov/}{U.S.\@ CDC.}} I also
      conducted
        {\href{https://www.sciencedirect.com/science/article/pii/S2589537021000146}{Bayesian
            analyses}} to estimate antibody deccay rates after SARS-CoV-2
      infection.}

  }

}

\spacedhrule{-0.2em}{-0.4em}

\roottitle{Education}

\headedsection
{University of Cambridge}
{\textsc{Cambridge, UK}} {%
  \headedsubsection
  {Ph.D. Infectious Disease Informatics, Department of Zoology} {2014 -- 2019}
  {\bodytext{Supervised by Prof.\@ Derek Smith. Quantifying the relationship
      between vaccine effectiveness and antigenic mismatch. Developing linear mixed
      models for association testing and genotype to phenotype mapping with
      influenza virus antigenic phenotypes. Developing and testing a framework to
      rank substitutions by their similarity to substitutions that have caused
      antigenic cluster transitions.} }
}

\headedsection
{Imperial College London}
{\textsc{London, UK}} {%
  \headedsubsection
  {MRes Biosystematics, Natural History Museum}
  {2012 -- 2013}
  {\bodytext{
      I conducted three independent research projects
      \begin{itemize}
        \item Endogenous retrovirus screening in catarrhine primates. Supervised by Dr.\@ Michael Tristem.
        \item Novel methods in mitochondrial DNA enrichment. Supervised by Dr.\@ Martijn Timmermans.
        \item A morphometric assessment of species delimitation in Canarian Pericallis. Supervised by Dr.\@ Mark Carine.
      \end{itemize}
    }
  }
}

\headedsection
{University of Cambridge}
{\textsc{Cambridge, UK}} {%
  \headedsubsection
  {BA Natural Science, Queens' College}
  {2009 -- 2012}
  {
    \bodytext{
      \begin{itemize}
        \item Part II Zoology. Research Project: Combining molecular and
              morphological data in phylogenetic analyses. Supervised by Dr.\@ Robert
              Asher.
        \item Part IB Biochemistry, Animal Biology, Plant Biology.
        \item Part IA Evolution and Behaviour, Biology of Cells, Chemistry,
              Mathematics for Biologists.
      \end{itemize}
    }
  }
}

\headedsection
{Aylesbury Grammar School}
{\textsc{Aylesbury, UK}} {
  \headedsubsection
  {A level Maths, Further Maths, Biology, Chemistry, Physics}
  {2000 -- 2008}
  {}
}

\spacedhrule{0.5em}{-0.4em}

\roottitle{Skills}

\inlineheadsection  % special section that has an inline header with a 'hanging' paragraph
{Computational:} {I have used \textbf{python} daily for 9 years; developing
  (e.g.\@ \href{https://ititer.readthedocs.io/}{ititer},
  \href{https://pymds.readthedocs.io}{pymds}) and using scientific research
  packages (e.g. \textbf{pymc3}, \textbf{numpy}, \textbf{scipy},
  \textbf{scikit-learn}, \textbf{matplotlib}, \textbf{pandas}, \textbf{bambi}).
  I make apps with \textbf{dash} and have some experience with \textbf{django}.
  \sbull I am a big proponent of \textbf{unit testing} for code I write and
  datasets I curate. \sbull I am familiar with \textbf{R} and
  \textbf{Javascript}. \sbull I use the unix command line and \textbf{git}
  daily. \sbull Bioinformatics software I have used includes: \textbf{MrBayes},
  \textbf{RAxML} and \textbf{mafft} for phylogenetics and \textbf{gromacs} and
  \textbf{amber} for structural biology, among many others. \sbull I have used
  \textbf{snakemake} in several projects.}

\vspace{0.5em}
\inlineheadsection
{Natural languages:}
{English \emph{(mother tongue)} and German\emph{(elementary proficiency)}.}


\spacedhrule{1.6em}{-0.4em}

\roottitle{Publications}

\inlineheadsection
{}
{
  \renewcommand{\bibsection}{(See this list on {\href{https://scholar.google.co.uk/citations?user=q260RVcAAAAJ&hl=en}{Google scholar.)}}}
  \nocite{*}
  \bibliography{ref}
}

\spacedhrule{1.6em}{-0.4em}

% \roottitle{Interests}

% \inlineheadsection
%   {Non-exhaustive and in alphabetical order:}
%   {art, Buddhism, cryptography, functional programming, Go (board game), history, music (from classical and jazz to Berlin-techno), \acr{NLP}, permaculture, philosophy, rock climbing, startups, travel, typography (e.g.\ graphic design, \LaTeX), \acr{UX}-design and vegan cuisine.}


\end{document}

