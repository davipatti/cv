\documentclass[10pt,a4paper]{article}

\usepackage[a4paper,margin=0.7in]{geometry}

\usepackage[
    sorting=ydnt,
    maxnames=50,
    backend=biber,
    style=authoryear
]{biblatex}
\addbibresource{ref.bib}

\DeclareBibliographyCategory{projects}

% Slightly more space between bib items
\setlength{\bibitemsep}{1.5\itemsep}

% A bit more space around bib notes
\DeclareFieldFormat{note}{\vspace{0.5em}\par\small#1}

% Don't print "In: journal", just print "journal".
\renewbibmacro{in:}{%
  \ifentrytype{article}{}{%
    \printtext{\bibstring{in}\intitlepunct}%
  }%
}

\DeclareNameFormat{family}{%
  \namepartfamily
  \usebibmacro{name:andothers}%
}

\usepackage[english]{babel}

% Makes resume-specific commands available.
\usepackage{resume}

\begin{document}
\sloppy  % this to relax white spacing in favour of straight margins

\maintitle{David J. Pattinson}{September 19, 1989}{Updated \today}

\nobreakvspace{0.3em}  % add some page break averse vertical spacing

% \noindent prevents paragraph's first lines from indenting
% \mbox is used to obfuscate the email address
% \sbull is a spaced bullet
% \href well..
% \\ breaks the line into a new paragraph
\noindent\href{mailto:david.pattinson@wisc.edu}{david.pattinson\mbox{}@\mbox{}wisc.edu}\sbull
\textsmaller{+1} (516) 413--4078\sbull
\href{https://github.com/davipatti}{https://github.com/davipatti}
\par \noindent 1 Bungtown Rd, Cold Spring Harbor, NY 11724, USA

\spacedhrule{0.5em}{-0.4em}  % a horizontal line with some vertical spacing before and after

\roottitle{Summary} \begin{multicols}{2}

  \emph{ I am a computational biologist with a background in evolutionary biology and
  infectious diseases. My research spans inferring infection histories from messy
  serological data, elucidating the antigenic effect of mutations and testing the
  predictability of seasonal influenza viruses. I am a Scientist at the Influenza
  Research Institute, University of Wisconsin---Madison, where I lead computational
  investigations in the group, teach, and run my own projects. In my career I have
  developed strong skills in Bayesian inference, high throughput computing, data
  visualization and phylogenetics. }

  \emph{ Fundamentally, I am interested in how different visual or quantitative
  representations of data drive scientific inquiry. Data
  visualization is a core component of my work. I often iterate to generate custom
  visualizations that provide detailed insight into datasets and the workings of
  statistical models while empowering the viewer to see patterns they did not know they
  were looking for. Quantitatively, I have used antigenic cartography,
  antibody landscapes and custom Bayesian models to resolve antigenic variation
  of viruses, infer the effects of mutations, and uncover predictable patterns
  in evolution. }
\end{multicols}

\spacedhrule{0.9em}{-0.4em}

\roottitle{Experience}

\headedsection
{Influenza Research Institute, {University of Wisconsin---Madison}}{\textsc{Madison WI, USA}}{

  \headedsubsection
  {Scientist II}
  {Mar \apo24--present}
  {

%    \bodytext{I spearhead computational investiagations in the group and guide their
%    strategic direction.}

  }

  \headedsubsection
  {Scientist I}  % https://hr.wisc.edu/standard-job-descriptions/?q=RE043
  {Mar \apo22--Mar \apo24}
  {

    \bodytext{ I lead computational investigations for the group which comprise antigenic
    cartography, antibody landscapes, phylogenetics, computational structural biology,
    next-generation sequencing analyses and constructing bespoke Bayesian models.}

  }

  \headedsubsection
  {Postdoctoral Research Associate}
  {Nov \apo19--Mar \apo22}
  {

    \bodytext{I joined Prof.\@ Yoshihiro Kawaoka's lab to research mechanisms of
      influenza virus antigenic change using computational structural biology. During the
      COVID-19 pandemic I helped run a COVID-19 serosurveillance cohort in collaboration
      with the {\href{https://www.cdc.gov/}{U.S.\@ CDC}} and the
      {\href{https://www.marshfieldresearch.org/}{Marshfield Clinic Research Institute}}.
      }

  }

}

\spacedhrule{0.9em}{-0.4em}

\roottitle{Education}

\headedsection
{University of Cambridge}
{\textsc{Cambridge, UK}} {%
  \headedsubsection
  {Ph.D. Infectious Disease Informatics, Department of Zoology} {2014--2019}
  {\bodytext{Supervised by Prof.\@ Derek Smith \sbull
  Quantifying the relationship between vaccine effectiveness and antigenic mismatch \sbull
  GWAS linear mixed models for association testing and genotype-phenotype mapping applied to influenza virus antigenicity \sbull
  A framework to predict influenza virus antigenic cluster transition substitutions.
  } } }

\headedsection
{Imperial College London}
{\textsc{London, UK}} {%
  \headedsubsection
  {\href{https://en.wikipedia.org/wiki/Master_of_Research}{M.Res.} Biosystematics, Natural History Museum}
  {2012--2013}
  {\bodytext{
        Endogenous retrovirus screening in catarrhine primates (Dr.\@ Michael Tristem) \sbull
        Novel methods in mitochondrial DNA enrichment (Dr.\@ Martijn Timmermans) \sbull
        A morphometric assessment of species delimitation in Canarian Pericallis (Dr.\@ Mark Carine).
    }
  }
}

\headedsection
{University of Cambridge}
{\textsc{Cambridge, UK}} {%
  \headedsubsection
  {B.A. Natural Science, Queens' College}
  {2009--2012}
  {
    \bodytext{
      Part II Zoology. Research Project: Combining molecular and
          morphological data in phylogenetic analyses (Dr.\@ Robert Asher) \sbull
      Part IB Biochemistry, Animal Biology, Plant Biology \sbull
      Part IA Evolution and Behaviour, Biology of Cells, Chemistry,
          Mathematics for Biologists.
    }
  }
}

\spacedhrule{0.9em}{-0.4em}

\roottitle{Skills}

\inlineheadsection  % special section that has an inline header with a 'hanging' paragraph
{Computational:} {

  I have used \textbf{Python} on a daily basis since 2013; developing (e.g.\@
  \href{https://ititer.readthedocs.io/}{ititer},
  \href{https://pymds.readthedocs.io}{pymds}) and using scientific research packages
  (e.g. \textbf{PyMC}, \textbf{NumPy}, \textbf{MDAnalysis}, \textbf{SciPy},
  \textbf{scikit-learn}, \textbf{matplotlib}, \textbf{pandas}, \textbf{Bambi}) \sbull I
  develop interactive visualization dashboards with \textbf{dash} \sbull I am a proponent of
  \textbf{unit testing} and \textbf{reproducible research practices} \sbull I am familiar
  with \textbf{R} and \textbf{JavaScript}, and am learning \textbf{Rust} \sbull I use the
  \textbf{UNIX} command line and \textbf{git} daily \sbull Bioinformatics software I have
  used includes: \textbf{MrBayes}, \textbf{RAxML} and \textbf{MAFFT} for phylogenetics
  and \textbf{GROMACS} and \textbf{Amber} for structural biology \sbull I have used
  \textbf{Snakemake} in several projects \sbull I have used the \textbf{SLURM} and
  \textbf{HTCondor} cluster scheduling systems.}

\vspace{0.5em}
\inlineheadsection
{Languages:}
{English \emph{(mother tongue)}, German\emph{(elementary)}.}


\nocite{*}

\addtocategory{projects}{pattinson.2024.2,pattinson.2024.3,pattinson.2024.4}

\printbibliography[title=Current Projects,category=projects]

\printbibliography[title=Publications,notcategory=projects]

\end{document}
