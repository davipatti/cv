%%%%%%%%%%%%%%%%%%%%%%%%%%%%%%%%%%%%%%%%%%%%%%%%%%%%%%%%%%%%%%%%%%%%%%%%%%%%%%%%
% Medium Length Graduate Curriculum Vitae
% LaTeX Template
% Version 1.2 (3/28/15)
%
% This template has been downloaded from:
% http://www.LaTeXTemplates.com
%
% Original author:
% Rensselaer Polytechnic Institute
% (http://www.rpi.edu/dept/arc/training/latex/resumes/)
%
% Modified by:
% Daniel L Marks <xleafr@gmail.com> 3/28/2015
%
% Important note:
% This template requires the res.cls file to be in the same directory as the
% .tex file. The res.cls file provides the resume style used for structuring the
% document.
%
%%%%%%%%%%%%%%%%%%%%%%%%%%%%%%%%%%%%%%%%%%%%%%%%%%%%%%%%%%%%%%%%%%%%%%%%%%%%%%%%

%-------------------------------------------------------------------------------
%	PACKAGES AND OTHER DOCUMENT CONFIGURATIONS
%-------------------------------------------------------------------------------

%%%%%%%%%%%%%%%%%%%%%%%%%%%%%%%%%%%%%%%%%%%%%%%%%%%%%%%%%%%%%%%%%%%%%%%%%%%%%%%%
% You can have multiple style options the legal options ones are:
%
%   centered:	the name and address are centered at the top of the page
%				(default)
%
%   line:		the name is the left with a horizontal line then the address to
%				the right
%
%   overlapped:	the section titles overlap the body text (default)
%
%   margin:		the section titles are to the left of the body text
%
%   11pt:		use 11 point fonts instead of 10 point fonts
%
%   12pt:		use 12 point fonts instead of 10 point fonts
%
%%%%%%%%%%%%%%%%%%%%%%%%%%%%%%%%%%%%%%%%%%%%%%%%%%%%%%%%%%%%%%%%%%%%%%%%%%%%%%%%

\documentclass[margin]{res}

% Default font is the helvetica postscript font
\usepackage{helvet}
\usepackage{hyperref}
\usepackage{biblatex}[backend=biblatex,style=science]
\addbibresource{references.bib}


% Increase text height
\textheight=700pt

\begin{document}

%-------------------------------------------------------------------------------
%	NAME AND ADDRESS SECTION
%-------------------------------------------------------------------------------
\name{David Joseph Pattinson | BA MRes}
\address{e: davipatti10[at]gmail.com\\t: +44 7817 188201\\\url{https://github.com/davipatti}\\dob: 19\textsuperscript{th} Sep 1989}
\address{Centre for Pathogen Evolution\\Department of Zoology\\University of Cambridge\\Cambridge, CB2 3EJ, UK}
%-------------------------------------------------------------------------------

\begin{resume}


%-------------------------------------------------------------------------------
%	EDUCATION SECTION
%-------------------------------------------------------------------------------
\section{Education}
\textbf{University of Cambridge}, Cambridge, UK\\
{\sl PhD Infectious disease}. Department of Zoology, September 2014\hfill Ongoing\\
{\sl BA Natural Science}. Queens' College, 2009--2012\hfill 1\textsuperscript{st} class

\textbf{Imperial College London}, London, UK\\
{\sl MRes Biosystematics}. Natural History Museum, 2012--2013\hfill Distinction

\textbf{Alyesbury Grammar School}, Aylesbury, UK\\
{\sl A level Maths, Further Maths, Biology, Chemistry, Physics}. 2000--2008\hfill 5$\times$A
%-------------------------------------------------------------------------------

%-------------------------------------------------------------------------------
%	COMPUTER SKILLS SECTION
%-------------------------------------------------------------------------------
\section{Computational\\skills}

I have used \textbf{Python} daily for 5 years; I write (e.g. \url{https://pymds.readthedocs.io}) and use packages for scientific research (e.g. \textbf{numpy}, \textbf{scipy}, \textbf{sci-kit learn}, \textbf{matplotlib}, \textbf{pandas}.) I have done some app development in \textbf{dash}.
|
I am familiar with \textbf{R} and \textbf{Javascript}.
|
I use the unix command line daily.
|
Scientific software I have used includes: \textbf{MrBayes}, \textbf{RAxML} and \textbf{mafft} for phylogenetics and \textbf{gromacs} and \textbf{amber} for structural biology.

%-------------------------------------------------------------------------------
%	PROJECTS SECTION
%-------------------------------------------------------------------------------
\section{Research\\projects}

\par
\textbf{Analysis of the antigenic evolution of seasonal influenza viruses}: Quantifying the relationship between VE and mismatch | Applying linear mixed models for association testing and genotype to phenotype mapping with antigneic phenotypes | A framework to rank substitutions by risk of causing major antigenic change.\\
{\sl PhD} \hfill Supervised by Prof. Derek Smith \hfill Ongoing

\textbf{Endogenous retrovirus screening in catarrhine primates}\\
{\sl MRes} \hfill Supervised by Dr. Michael Tristem\hfill Distinction

\textbf{Novel methods in mitochondrial DNA enrichment}\\
{\sl MRes} \hfill Supervised by Dr. Martijn Timmermans\hfill Distinction

\textbf{A morphometric assessment of species delimitation in Canarian Pericallis}\\
{\sl MRes}\hfill Supervised by Dr. Mark Carine\hfill Distinction\\
This contributed to a publication: \url{https://doi.org/10.1186/s12862-016-0766-1}.

\textbf{Combining molecular and morphological data in phylogenetic analyses}\\
{\sl BA} \hfill Supervised by Dr. Robert Asher\hfill 1\textsuperscript{st} class.\\
For this project I won the {\sl Palaeontological Association Undergraduate Prize} and {\sl John Ray Trust Science Prize}.\\
This project resulted in the following publication: \url{https://doi.org/10.1093/sysbio/syu077}.

%-------------------------------------------------------------------------------

%-------------------------------------------------------------------------------
%	EXPERIENCE SECTION
%-------------------------------------------------------------------------------
% Modify the format of each position
\begin{format}
\title{l}\employer{r}\\
\dates{l}\location{r}\\
\body\\
\end{format}
%-------------------------------------------------------------------------------

\section{Additional\\experience}
\employer{Nautral History Museum}
\location{London, UK}
\dates{Oct 2013 - Apr 2014}
\title{\textbf{Research assistant}}
\begin{position}
    Compiled information for and helped develop Hypericum online: \url{http://hypericum.myspecies.info/}.
\end{position}

\employer{University Museum of Zoology}
\location{Cambridge, UK}
\dates{Jun--Aug 2011 and 2012}
\title{\textbf{Research internships}}
\begin{position}
    Phylogenetic analyses using combined morphological and molecular data.
    |
    Artificial extinction experiments.
    |
    Characterised prenatal dental eruption sequences using $\mu$CT imagery.
    |
    Funded by the Weis-Fogh and the J. Arthur Ramsay funds.
    |
    Contributed to this publication: \url{https://doi.org/10.1080/02724634.2017.1317638}.
\end{position}

%-------------------------------------------------------------------------------
\end{resume}
\end{document}
