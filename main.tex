%%%%%%%%%%%%%%%%%%%%%%%%%%%%%%%%%%%%%%%%%%%%%%%%%%%%%%%%%%%%%%%%%%%%%%%%%%%%%%%%
% Medium Length Graduate Curriculum Vitae
% LaTeX Template
% Version 1.2 (3/28/15)
%
% This template has been downloaded from:
% http://www.LaTeXTemplates.com
%
% Original author:
% Rensselaer Polytechnic Institute
% (http://www.rpi.edu/dept/arc/training/latex/resumes/)
%
% Modified by:
% Daniel L Marks <xleafr@gmail.com> 3/28/2015
%
% Important note:
% This template requires the res.cls file to be in the same directory as the
% .tex file. The res.cls file provides the resume style used for structuring the
% document.
%
%%%%%%%%%%%%%%%%%%%%%%%%%%%%%%%%%%%%%%%%%%%%%%%%%%%%%%%%%%%%%%%%%%%%%%%%%%%%%%%%

%-------------------------------------------------------------------------------
%	PACKAGES AND OTHER DOCUMENT CONFIGURATIONS
%-------------------------------------------------------------------------------

%%%%%%%%%%%%%%%%%%%%%%%%%%%%%%%%%%%%%%%%%%%%%%%%%%%%%%%%%%%%%%%%%%%%%%%%%%%%%%%%
% You can have multiple style options the legal options ones are:
%
%   centered:	the name and address are centered at the top of the page
%				(default)
%
%   line:		the name is the left with a horizontal line then the address to
%				the right
%
%   overlapped:	the section titles overlap the body text (default)
%
%   margin:		the section titles are to the left of the body text
%
%   11pt:		use 11 point fonts instead of 10 point fonts
%
%   12pt:		use 12 point fonts instead of 10 point fonts
%
%%%%%%%%%%%%%%%%%%%%%%%%%%%%%%%%%%%%%%%%%%%%%%%%%%%%%%%%%%%%%%%%%%%%%%%%%%%%%%%%

\documentclass[margin]{res}

\topmargin=-0.5in % start text higher on the page
\oddsidemargin=-0.5in
\textwidth=6in % increase textwidth to get smaller right margin
\usepackage{hyperref}
\hypersetup{colorlinks=true,linkcolor=blue,urlcolor=blue}

% Increase text height
\textheight=700pt

\begin{document}

%-------------------------------------------------------------------------------
%	NAME AND ADDRESS SECTION
%-------------------------------------------------------------------------------
\name{David Joseph Pattinson | Ph.D. MRes BA}
\address{
	e: david.pattinson[at]wisc.edu\\
	t: +1 (516) 413 4078\\
	dob: 19\textsuperscript{th} Sep 1989\\
	\url{https://github.com/davipatti}
}
\address{
	Influenza Research Institute\\
	Science Dr.\\
	Madison, WI 53711\\
	USA
}
%-------------------------------------------------------------------------------



\begin{resume}

	I am a computational scientist interested in understanding, and predicting,
	the antigenic evolution of seasonal influenza viruses and tracking
	population immune responses to SARS-CoV-2. I use antigenic cartography,
	antibody landscapes, phylogenetics, computational structural biology,
	Bayesian inference and data visualisation.

	%-------------------------------------------------------------------------------
	%	Employment
	%-------------------------------------------------------------------------------
    \section{Employment}
    \textbf{University of Wisconsin--Madison}, Madison WI, USA\\
    {\sl Scientist}. Influenza Research Institute, Mar 2022--present\\
    {\sl Postdoctoral Research Associate}. Influenza Research Institute, Nov 2019--Mar 2022

	%-------------------------------------------------------------------------------
	%	EDUCATION SECTION
	%-------------------------------------------------------------------------------
	\section{Education}
	\textbf{University of Cambridge}, Cambridge, UK\\
	{\sl Ph.D. Infectious disease informatics}. Department of Zoology, 2014--2019\hfill Pass\\
	{\sl BA Natural Science}. Queens' College, 2009--2012\hfill 1\textsuperscript{st} class\\
	\textbf{Imperial College London}, London, UK\\
	{\sl MRes Biosystematics}. Natural History Museum, 2012--2013\hfill Distinction\\
	\textbf{Alyesbury Grammar School}, Aylesbury, UK\\
	{\sl A level Maths, Further Maths, Biology, Chemistry, Physics}. 2000--2008\hfill 5$\times$A
	%-------------------------------------------------------------------------------

	%-------------------------------------------------------------------------------
	%	COMPUTATIONAL SKILLS
	%-------------------------------------------------------------------------------
	\section{Computational\\skills}

	I have used \textbf{python} daily for 9 years; developing (e.g.
	\href{https://ititer.readthedocs.io/}{ititer},
	\href{https://pymds.readthedocs.io}{pymds}) and using scientific research
	packages (e.g. \textbf{pymc3}, \textbf{numpy}, \textbf{scipy},
	\textbf{scikit-learn}, \textbf{matplotlib}, \textbf{pandas},
	\textbf{bambi}.) I make apps with \textbf{dash} and have some experience
	with \textbf{django}. | I am familiar with \textbf{R} and
	\textbf{Javascript}. | I use the unix command line daily. | Scientific
	software I have used includes: \textbf{MrBayes}, \textbf{RAxML} and
	\textbf{mafft} for phylogenetics and \textbf{gromacs} and \textbf{amber} for
	structural biology, among many others. | I have used \textbf{snakemake} in
	several projects.
	
	%-------------------------------------------------------------------------------
	%	PROJECTS SECTION
	%-------------------------------------------------------------------------------
	\section{Research}
	\href{https://scholar.google.co.uk/citations?user=q260RVcAAAAJ&hl=en}{Link to my publications}.\\
	\textbf{Predicting the antigenic evolution of seasonal influenza viruses with application to vaccination strategy}.
	Quantifying the relationship between VE and mismatch.
	|
	Developing linear mixed models for association testing and genotype to phenotype mapping with antigenic phenotypes.
	|
	A framework to rank substitutions by similarity to cluster transition substitutions.
    |
    \href{https://www.repository.cam.ac.uk/handle/1810/304119}{Link to thesis}.
    \\
	{\sl Ph.D.} \hfill Supervised by Prof. Derek Smith \hfill Pass\\
	\textbf{Endogenous retrovirus screening in catarrhine primates}\\
	{\sl MRes} \hfill Supervised by Dr. Michael Tristem\hfill Distinction\\
	\textbf{Novel methods in mitochondrial DNA enrichment}\\
	{\sl MRes} \hfill Supervised by Dr. Martijn Timmermans\hfill Distinction\\
	\textbf{A morphometric assessment of species delimitation in Canarian Pericallis}\\
	{\sl MRes}\hfill Supervised by Dr. Mark Carine\hfill Distinction\\	
	\textbf{Combining molecular and morphological data in phylogenetic analyses}\\
	{\sl BA} \hfill Supervised by Dr. Robert Asher\hfill 1\textsuperscript{st} class.\\
	This won the Palaeontological Association Undergraduate Prize and John Ray Trust Science Prize.

	%-------------------------------------------------------------------------------

	%-------------------------------------------------------------------------------
	%	EXPERIENCE SECTION
	%-------------------------------------------------------------------------------
	% Modify the format of each position
	\begin{format}
		\title{l}\employer{r}\\
		\dates{l}\location{r}\\
		\body\\
	\end{format}
	%-------------------------------------------------------------------------------

	\section{Additional\\experience}
	\employer{Natural History Museum}
	\location{London, UK}
	\dates{Oct 2013 - Apr 2014}
	\title{\textbf{Research assistant}}
	\begin{position}
		Developed Hypericum online: \url{http://hypericum.myspecies.info/}.
	\end{position}
	\employer{University Museum of Zoology}
	\location{Cambridge, UK}
	\dates{Jun--Aug 2011 and 2012}
	\title{\textbf{Research internships}}
	\begin{position}
		Phylogenetic analyses using combined morphological and molecular data.
		|
		Artificial extinction experiments.
		|
		Characterised prenatal dental eruption sequences using $\mu$CT imagery.
		|
		Funded by the Weis-Fogh and the J. Arthur Ramsay funds.
	\end{position}

	%-------------------------------------------------------------------------------
\end{resume}

\end{document}
